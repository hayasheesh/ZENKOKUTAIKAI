\documentclass[twocolumn,9pt,a4paper,dvipdfmx]{jsarticle}

% --- パッケージ設定 ---
\usepackage[utf8]{inputenc}
\usepackage[T1]{fontenc}
\usepackage{mathptmx} % 英数字をTimes系に
\usepackage{float}               % [H]で図を強制配置
\usepackage[top=30mm, bottom=27mm, left=18mm, right=18mm, columnsep=7mm]{geometry}
\usepackage[dvipdfmx]{graphicx}
\usepackage{secdot}
\usepackage{caption}
\usepackage{titlesec}
\usepackage{cite}

% --- 図表と本文の間隔設定 ---
\setlength{\textfloatsep}{0pt} % ページ上下の図表と本文の間隔
\setlength{\intextsep}{0pt}    % 本文中の図表(h)と本文の間隔
\setlength{\floatsep}{0pt}     % 図表同士の間隔

% --- キャプション設定 ---
\captionsetup[figure]{font=small, labelsep=space, name=図, skip=3pt}
\captionsetup[table]{font=small, labelsep=space, name=表, skip=3pt}

% --- セクション設定 ---
\sectiondot{section}
\sectiondot{subsection}

% セクション:
% サイズ10pt, 行送り20pt
\titleformat{\section}
  {\fontsize{10pt}{20pt}\selectfont\mcfamily} 
  {\thesection}{1em}{}
% 上下の余白を0ptに設定
\titlespacing{\section}{0pt}{0pt}{0pt}

% サブセクション:
% サイズ9pt, 行送り14pt
\renewcommand{\thesubsection}{$<$\thesection$\cdot$\arabic{subsection}$>$}
\titleformat{\subsection}[runin]
  {\fontsize{9pt}{14pt}\selectfont\mcfamily}
  {\thesubsection}{1em}{}
% 上下の余白を0ptに設定
\titlespacing{\subsection}{0pt}{0pt}{1em}

% 参考文献形式 (1)
\renewcommand{\citeleft}{(}
\renewcommand{\citeright}{)}
\makeatletter
\renewcommand{\@biblabel}[1]{(#1)}
\makeatother

% --- 文書情報 ---
\date{}

\begin{document}

% --- タイトル部分 ---
\twocolumn[
    \begin{center}
        \vspace*{10mm}
        
        \parbox{134mm}{
            \centering
            \fontsize{18pt}{28pt}\selectfont \mcfamily 
            電気学会全国大会講演論文の書き方
        }
        
        \vspace{18pt}

        {\fontsize{12pt}{18pt}\selectfont \mcfamily
        研究 花子\textsuperscript{*}, 電気 太郎, 学会 次郎 (○○○大学) \par}

        \vspace{14pt}

        {\fontsize{9pt}{14pt}\selectfont
        Preparation of Papers for National Convention of I.E.E. JAPAN \par
        Hanako Kenkyu, Taro Denki, Jiro Gakkai (○○○University) \par}
        
        \vspace{14pt}
    \end{center}
]

% --- 本文開始 ---
\fontsize{9pt}{14pt}\selectfont

\section{まえがき}
発表論文原稿は,A4原寸で印刷されます。執筆の時は以下の説明をよく読んだ上で,お使いのワードプロセッサ等で可能な範囲で指示に従って原稿をお書きください。なお,この説明書は,講演論文のレイアウトの見本になっていますので,参考にしてください。

\section{12年大会からの変更点}
刷り上りの論文集の体裁は従来通りですが,以下の点が変更になっていますので注意してください。
\begin{itemize}
    \item 15年大会からPDF投稿を可能としました。
    \item 紙面投稿の場合専用の原稿用紙はありません。お手持ちのA4判白色の上質紙に印刷してください。
    \item 提出いただいた原稿は,CD-ROMを作成する際の原版としても使用します。CD-ROMはカラーが可能ですが,印刷論文集は白黒となります。ただし黄色などは印刷時に出力されないので,著者が白黒でプリントアウトして確認して下さい。
\end{itemize}

\section{レイアウトと文字サイズ}

\subsection{マージンとカラム幅}
原稿用紙のマージンおよびカラム幅(全ページ共通)は,表\ref{tab:margins}のとおりです。特に上下左右のマージンは厳守してください。

2カラム(2段組)とし,各コラムの幅,カラム間マージンは表\ref{tab:margins}のとおりです。本文の字詰は,1行あたり26文字程度とします。分量は,図面,写真等を含めて1枚ないし2枚,シンポジウムは4枚以内です。

\begin{table}[ht]
    \centering
    \caption{マージン\\Table 1. Margins}
    \label{tab:margins}
    \begin{tabular}{lc}
        \hline
        上マージン & 30mm \\
        下マージン & 27mm \\
        左右マージン & 18mm \\
        カラム間マージン & 7mm \\
        カラム幅 & 83.5mm \\
        \hline
    \end{tabular}
\end{table}

\begin{table}[ht]
    \centering
    \caption{文字サイズ\\Table 2. Type sizes}
    \label{tab:typesizes}
    \begin{tabular}{lcc}
        \hline
         & サイズ & 行送り \\
        \hline
        論文タイトル & 18pt & 28pt \\
        著者名 & 12pt & 18pt \\
        英文タイトル著者所属名 & 9pt & 14pt \\
        章タイトル & 10pt & 20pt \\
        本文 & 9pt & 14pt \\
        参考文献 & 8pt & 12pt \\
        \hline
    \end{tabular}
\end{table}

\subsection{配置}
表題等は,この見本に従って次の1.~4.の順序で記載し,本文を書き始めてください。(2ページ目以降は,1.~3.不要)文字サイズと行送りは,表\ref{tab:typesizes}を参考にしてください。

1. 表題:第1行中央に2カラム通しで書く(長ければ第2行も使う。第1行で済めば,第2行目は詰める)。表題1行目の左に,講演番号のスペースをあける。(テンプレートにおいては講演番号のスペースは設定してある)

2. 著者名および勤務先:表題の下を1行あけて,次の行から中央に2カラム通しで書く。講演者名の右肩に「\textsuperscript{*}」印を付ける。

3. 英文表題,氏名(所属):著者名および勤務先の下を1行あけて,次の行から中央に2カラム通しで書く。

4. 本文:英文による表題,氏名の下を1行あけて,次の行から書く。2ページは,上マージンに続いて第1行から本文を書く。

\subsection{文献}
文献は本文末尾に通し番号を付けて一括記載し,本文中の該当個所に引用番号を付けてください。文献の記載方法は,著者名,雑誌名,ページ,発行年の順序にしてください。
(ex: \cite{Shahzadi1965}, \cite{Amano1975}と書くと文献番号が挿入できます。)

\subsection{式および図}
式および図は,図\ref{fig:example}および以下の記載例を参考にしてください。図面等を貼り付ける場合は,しわにならないように注意してください。また,図および表の説明には,英文を併記してください。

% --- 図の例 ---
\begin{figure}[H]
    \centering
    \includegraphics[width=80mm]{images/fig1.png}
    \caption{図面の例\\Fig. 1. An example of figures}
    \label{fig:example}
\end{figure}

% --- 式の例 ---
\begin{equation}
    E = RI
\end{equation}
\begin{equation}
    V = Ri + L \frac{di}{dt}
\end{equation}

% --- 参考文献 ---
\begingroup
\fontsize{8pt}{12pt}\selectfont
\renewcommand{\refname}{文献}
\bibliographystyle{junsrt}
\bibliography{references}
\endgroup

\end{document}